

\usetikzlibrary {arrows.meta,automata,positioning,shadows,decorations.pathmorphing, decorations.pathreplacing, decorations.shapes, graphs, calc, math}
\usetikzlibrary{shapes.geometric}
\usetikzlibrary{decorations.pathreplacing}
\usetikzlibrary {arrows.meta}


\tikzset{%
   dot/.style={
      circle,
      inner sep=0mm,
      outer sep=0mm,
      minimum size=2mm,
      draw=black,
      fill=black
   },
    triple/.style={
      double distance=2pt,
      postaction={draw}
    },
    quadruple/.style={
      double,
      double distance=2pt,
      postaction={
        draw,
        transform canvas={yshift=-.4pt},
      },
      postaction={
        draw,
        transform canvas={yshift=.4pt},
      }
    },
   every loop/.style={},
   d/.style={
     Circle[]-,
     shorten <=-2pt,
     transform canvas={shift={(-3pt, 4pt)}}
   },
   d0/.style={
     Circle[]-,
     shorten <=-2pt,
   },
   d1/.style={
     Circle[]-,
   },
   dn/.style={
     draw, circle, minimum size=1.3em
   },
   ddn/.style={
     draw, circle, double, minimum size=1.3em
   },
   dddn/.style={
     draw, circle, double, minimum size=1.3em,
       double distance=2pt,
       postaction={draw}
   },
   triangle/.style={
    regular polygon,
    regular polygon sides=3,
    rotate=270,
    scale=.5,
    inner sep=3pt,
    draw
   },
}



\newcommand{\trace}[2]{% #1 is the list of items, #2 is the looseness for the loop
   %\pgfmathsetmacro{\negspace}{-1.0em - 0.5em*#2} % Calculate negative space
   \hspace{-1em}
    \begin{tikzpicture}[baseline=(node1.base), inner sep=1pt]
        % Initialize a counter for tracking the number of items
        \newcount\itemcount
        \itemcount=0
        
        % Define nodes
        \def\lastnode{node1}
        \foreach \i [count=\c] in {#1} {
            \ifnum \c=1
                \node (node\c) {$\i$}; % First node
            \else
                \node[right=1em of \lastnode] (node\c) {$\i$}; % Subsequent nodes
                \draw (\lastnode) -- (node\c); % Draw edge from last node to current
            \fi
            \global\advance\itemcount by 1
            \xdef\lastnode{node\c}
        }
        
        % Draw the loop edge
        \ifnum \itemcount=1
            \path (node1) edge [out=160, in=20, loop] ();
        \else
            \path (node1) edge [out=160, in=20, looseness=#2] (\lastnode);
        \fi
    \end{tikzpicture}
   \hspace{-1em}
}

\def\matmul#1{
   \vecmatvec{.5em}{}{#1}{}
}

\def\vecmatvec#1#2#3#4{
   \begin{tikzpicture}[baseline=-.25em, inner sep=1pt]
      \node (node0) {$#2$};
      \xdef\lastnode{node0};
      \foreach \i [count=\c] in {#3} {
         \node[right=#1 of \lastnode] (node\c) {$\i$};
         \draw (\lastnode.east) -- (node\c);
         \xdef\lastnode{node\c};
      }
      \node[right=#1 of \lastnode] (last) {$#4$};
      \draw (\lastnode.east) -- (last);
   \end{tikzpicture}
}

\def\detstack#1{
   \mathbin{\begin{tikzpicture}[baseline=(a0.base), inner sep=1pt]
      \node (a0) {#1};
      \node[right=.5em of a0] (dots) {$\cdots$};
      \node[right=.5em of dots] (a1) {#1};
      \draw (a0.north) -- ++(0,.2) coordinate (a0top);
      \draw (a1.north) -- ++(0,.2) coordinate (a1top);
      \draw (a0.south) -- ++(0,-.2) coordinate (a0bot);
      \draw (a1.south) -- ++(0,-.2) coordinate (a1bot);
      \draw[line width=2pt] (a0top -| a0.west) -- (a1top -| a1.east);
      \draw[line width=2pt] (a0bot -| a0.west) -- (a1bot -| a1.east);
   \end{tikzpicture}}
}

% Define a new command for drawing the ellipse
\NewDocumentCommand{\drawellipse}{m m m m m o}{
    \def\centerX{#1}
    \def\centerY{#2}
    \def\widthR{#3}
    \def\heightR{#4}
    \def\angle{#5}
    
    \draw (\centerX,\centerY) ellipse [x radius=\widthR, y radius=\heightR];
    \fill ({\centerX + \widthR*cos(\angle)},{\centerY + \heightR*sin(\angle)}) circle [radius=0.075];
    
    % If target node is provided, use it, otherwise fall back to default behavior
    \IfNoValueTF{#6}{
        \draw ({\centerX + \widthR*cos(\angle)},{\centerY + \heightR*sin(\angle)}) -- ({\centerX + .5 + \widthR*cos(\angle)}, {\centerY + \heightR*sin(\angle)});
    }{
        \draw ({\centerX + \widthR*cos(\angle)},{\centerY + \heightR*sin(\angle)}) -- (#6);
    }
}
